
The ATLAS run control (RC) allows a coordinated control of hardware and software
resources. The RC performs \emph{transition}, to which the resource reacts
performing specific operations, depending on setup configurations that can be
specified in a centralized service, the object key service (OKS). The RC commands
are interpreted as transition in a final state machine, all components are
expect to move to the same state responding to a request. Each item controlled
by the RC performs the transition without possibility to interact with other components.
In practice, to control a resource, in this case the AMB, there is the need of a 
\textbf{RC module} and at least one \textbf{OKS schema} file. The RC module describes
an objects that reacts to the specific transition of the RC FSM, while the OKS
schema represents the parameters that describe the RC module at runtime. Each 
AMB RC module has a related schema.

\subsection{Run control module implementation}
\label{sec:rcmodule}

The AMB RC module class is the \textbf{\RCModule}, inheriting
form the \texttt{ROS::ReadoutModule}. It reimplements the following
public methods: setup, configure, ... more details on the operations
performed in each commands are given in the following sub-sections.

\subsubsection{Setup}

This method reads the OKS configuration for the given \AMBoard, 
retrieving all the required keys. 

\subsubsection{Configure} 

The configure methods reacts to the \textbf{configure} transition of the RC,
and its performed as soon after the transition starts. The internal
operations are controlled by the RC\_configure high level procedure and
loads a pattern bank,
see page \pageref{sec:procconfigure} in \ref{sec:procconfigure}.

\subsection{The AMB schema files}

To describe an \AMBoard there are a few schema files that are available, 
it follow the documentation of all the attributes.

\subsubsection{ReadoutModule\_Ambslp schema file}

This file describes the parameters of \RCModule. It inherits from 
\texttt{ReadoutModule} and \texttt{ResourceSetAND}. It declares the following
attributes:
\begin{description}
\item[Name:] sets the module name, e.g. ``AMB-2-15''. No default value is provided.

\item[Crate:] a numeric identifier for the crate. Type u32, default: 0.

\item[Slot:] the slot number where the board is plugged. Type u32, default: 0.

\item[HITFWVersion:] firmware version expected for the HIT FPGA. If the
FW version doesn't match the configuration procedure fails. 
If the value is $0x0$ the check is not performed. Type u32, hexadecimal, 
default: 0

\item[ROADFWVersion:] firmware version expected for the ROAD FPGA. 
Type u32, hexadecimal, default: 0

\item[VMEFWVersion:] firmware version expected for the ROAD FPGA. 
Type u32, hexadecimal, default: 0

\item[CTRLFWVersion:] firmware version expected for the CTRL FPGA. 
Type u32, hexadecimal, default: 0


\item[LAMBFWVersion:] firmware version expected for the BUSCA FPGA each LAMB;
all installed LAMBs are expected to have the FW version. 
Type u8, hexadecimal,  default: 0.

\item[DryRun:] boolean flag, when true the RC module avoids to perform VME calls,
allowing the RC infrastructure to be tested even without boards. Type bool,
default: 0.

\item[ForceWrite:] boolean flag used to force the write of a pattern bank; when
false the bank checksum register is matched with the one of the bank in memory, then
patterns are then loaded only if the two don't match; when true the bank is realoded,
skipping the check. Type bool, default: 0.

\item[DummyHit:] this represents the value of the \emph{dummy hit} word registers,
if the value is $0x0000000$ the feature is disabled, if not a dummy hit loop is
started during IDLE periods, as explained in \ref{sec:powercontrol}. Type u32,
default: 0.

\item[DCDCOVerVolt:] this value is used by the DCDC setup configuration to reset
the DCDC value, the value is function of the bias resistor that usually depend
on the board version type but it may change for other reasons. Type u32, default
$0x18f6$.

\item[DCDCSetVolt:] this value is used to set the DCDC voltage output to the
desired value, as some relation with the resistors mounted on the DCDC. Type u32,
hexadecimal, default $0x02e3$ (1.15 V output on AMBv5).

\item[DCDCScale:] value used to convert the voltages obtained from the DCDC
check in volts. Type float, default $1.6666666666666667$.

\item[AddTestPattern:] boolean flag that control the writing of the test pattern,
assumed to exist in every chip (see Sec.~\ref{sec:chipconfig}). If true a test
pattern is forced in the chip's position 0, patterns loaded by the user will start from
position 1 in the chip. If true the available space for patterns will be
reduced of 64, because 1 special pattern per chip will added by the procedure.
If false the pattern bank from the user will be loaded as it is.

\item[OverrideSpecialPattern:] boolean flag that control the special pattern
SS values overwriting the SSs of the original bank, this implicitly assumes 
that the pattern bank has been prepared respecting this board constraint.
Type bool, default 1`.'

\item[DumpFolder:] path where dump files are saved. Currently unused.

\item[LAMBMask:] 32-bit hexadicimal mask, unused.

\item[InitialThreshold:] threshold used during the pattern bank loading. Type u32,
decimal, default 15.

\item[Threshold:] pattern matching threshold. Type u32, decimal, default 7.

\item[CheckBank:] flag that enables the possibility to check the bank content
through JTAG. The procedure is extremely slow and not performed by default.
Type bool, default 0.

\item[BypassBus:] unused. Type u32, hexadecimal, default 0x24.

\item[PCBVersion:] unused. Type u32, decimal, default 0.

\item[DVmajority:] unused. Type boolean, default 0.

\item[Extras:] array of strings, with dynamic size, that can be freely used by the code.
The key is provided to allow to have unexpected parameter in the code in all cases
where a new schema cannot be deployed, e.g. quick code patches. The parsing of the
field is expected to be strongly dependent from the code tag.

\item[PatternBankConf:] link to an OKS object of the \textbf{PatternBankConf} type,
see Sec.~\ref{sec:pattbankschema}. This objects describe the location of the pattern
bank to be loaded in this bank, if no object is provided the configuration stage
of the \RCModule will fail.

\item[AMBStandaloneTest:] standalone test object, defined in Sec.~\ref{sec:ambtestschema}.
When an object on this type is set the \AMBoard is expected to load test-vector
data from the HIT input FIFOs instead of expecting data from the AUX. If no
object is provided the test is not performed.
\end{description}


\subsubsection{PatternBankConf schema}
\label{sec:pattbankschema}

\subsubsection{AMBStandaloneTest schema}
\label{sec:ambtestschema}

This schema represents the input parameters for an \AMBoard standalone test.
This kind of tests reads data from the HIT FPGA input FIFOs instead of waiting
them from the AUX. During this tests, AUX data will be rejected, while the
output can reach the AUX.

\begin{description}
	\item[TestFilePath'':] this is the path of the file containing the data to
	be loaded into the board. The path has to be accessible from the RCD application,
 indeed	from a path available in SBC computers of the crate, where the board is used,
 or a network path, eventually public. Type string.
 
 \item[Loop:] boolean flag that controls if the data are sent only once or
 in a continuous loop. The loop has to stopped through RC or generally sending
 an init to the board. Type bool, default 0.
\end{description}