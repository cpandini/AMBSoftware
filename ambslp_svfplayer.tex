It is possible to upload firmware on the board using SVF files,
produced using Xilinx Impact or other software, in the board
using the VME interface. An intrinsic limitation of this procedure is
that the firmware of VME FPGA cannot be changed.

The procedure is based on the creation an SVF file, the SVF file can then
be used by a specific program that send uses those information to reprogram
the FPGAs: \texttt{\index{ambslp\_svfplayer}}. The SVF file an contain information
to reconfigure the FPGA, reprogramming the flash memories or other
common operations, as reading the ID codes.

\subsubsection{Create the SVF file using Xilinx Impact}
To create the SVF it is necessary to open the Xilin Impact software. In case
the computer is connected to an \AMBoard scan the cable and remove the 
first FPGA in the JTAG chain, this is the VME FPGA and is not available
within the JTAG chain seen during the remote programming procedure.

An Impact project with the right JTAG chain is available in the firmware
repository \href{https://svnweb.cern.ch/trac/atlasftkfw/browser/AMboard/trunk/AMBSLP/AMBv4_vmechain.ipf}{\texttt{\textbf{AMBv4\_vmechain.ipf}}} and can be opened in impact to generate the SVF
files.

After the chain is modified, and reflects the needs for remote programming,
the proper bitstream and flash files can be assigned to the components in the
chain. It is not necessary to edit all elements, the general rule is in fact to
provide a specific SVF file for any major FPGA configuration.

To create then an SVF file go to the menu 
``Output>SVF File'', then click on ``Create SVF File''. Choose a position and
a name for the output file, a dialog box with an information message will appear,
as well as a blue text on the top right corner of the impact. More messages on the bottom
status bar will repeat that Impact is in SVF mode. Perform then all the operations
you would do as if a regular cable is plugged. All output will be sent to the file,
instead to the actual devices. Multiple operations can be done in the same sessions.
When the all commands are sent go in the SVF file menu item and close the file.
window will appear confirming that the 
While the SVF file receives the commands multiple operations can be performed,
it is common to perform only a major operation per session, e.g. program a single
FPGA or a single SPI FLASH memory, read the ID codes\ldots.


\subsubsection{Remote programming \AMBoard and LAMBs}

The program to be used to perform the remote programming, or other operations
on the FPGA chain, for either the mainboard
or the LAMBs is \textbf{\texttt{\index{ambslp\_svfplayer}}}. The program has to
be executed in the SBC where the board to be programmed is installed and has
a very simple command line. Available options are:
\begin{description}
\item[--slot num:] set the slot number where the board to be configured is
plugged, default 15.

\item[--svffile, -S file:] this option sets the SVF file(s) to be executed to
configure the FPGAs. The option can be used multiple times, all files will be
executed, in the order they appear in the command.

\item[--lamb, -l:] if present the operations in the SVF file(s) will be sent
to the LAMBs.

\item[--bt, -B:] if present the communication with the board will use the 
VME block-transfer mechanism.

\item[--nocomment, -N:] if present the comments in the SVF file are not
showed in the output.

\item[--verbose, -v num:] set the default level, increasing or decreasing
the amount of output information, default 1.
\end{description}

An example of the use for the program is the following:
\begin{verbatim}
$ ambslp_svfplayer -S test_idcode_amb.svf
Begin of programming sequence for AMB
Comment:  Created using Xilinx Cse Software [ISE - 14.7]
Advance 1% read 50 of 2902
Total number of bits: 0
Total number of TDO errors: 0 (Tot checks 0)
Total RUNTEST IDLE time 0 secs, Tcks 0
Comment:  Date: Wed Oct 19 16:14:49 2016
...
Advance 99% read 2873 of 2902
Total number of bits: 661
Total number of TDO errors: 0 (Tot checks 9)
Total RUNTEST IDLE time 0 secs, Tcks 0
ALL GOOD
Switching off programming for AMB

Elapsed time: 0 seconds
\end{verbatim}
in this case the default slot is implied for the board and a simple
SVF file that reads, and verifies, the ID code of the installed FPGAs is used.
This operation is used during the board testing to verify a good 
communication between the VME FPGAs and the JTAG chain. 

The software performs a check of the TDO signals, allowing to verify
that the output messages from the JTAG commands in the SVF file are as
expected. An ALL GOOD message is visible in the final part of the program.
In case of mismatches in the TDOs are met, their number is reported, the
exit status of the program is 0 if all TDOs match the expectation, -1
otherwise.

SVF files generated by Impact usually contain comments, used to 
described specific steps of complex operations, i.e. during flash programming.
The player prints out those messages as they appear, allowing to monitor
how the advancement of the whole operation. Other messages that show
which fractions of the SVF file has been parsed are also shown, in this
case also the total number of bits sent to the JTAG chain is also 
printed, as well as a summary of the total TDO messages checked, and
eventual errors, until now.
this point 