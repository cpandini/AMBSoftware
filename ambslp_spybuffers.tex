An important tool to debug an \AMBoard while is working is to check the
spy-buffers content. A spy-buffer is a buffer that stores the last data words
received, or sent, by the board. In total there are 12 input and 16 output
spy-buffers. Three different softwares are available to obtain the spy-buffers
content: \textbf{\texttt{ambslp\_inp\_spy\_main}}\index{ambslp\_inp\_spy\_main}
 is used to retrieve
the input spy-buffers, \textbf{\texttt{ambslp\_out\_spy\_main}}
\index{ambslp\_out\_spy\_main}
collects the output spy-buffers content, 
the most recent one is 
\textbf{\texttt{ambslp\_spybuffer\_dumper\_main}}
\index{ambslp\_spybuffer\_dumper\_main}, able to read
both input and output buffers and it will only tool maintained in the future.
More details on each program will be given in the following subsections.

\subsubsection{Input spy-buffers dump}

The \textbf{\texttt{ambslp\_inp\_spy\_main}}\index{ambslp\_inp\_spy\_main}
command
allows to obtain the content of the 12 input spy-buffers. The program 
requires a limited amount of options:
\begin{description}
\item[--slot num:] the slot number where the board is installed.

\item[--method num:] this allows to change how the spy-buffers content is presented,
more details later. Default is 0.
\end{description}

When used, the default behavior is to write on the screen a big table with 8192 rows
and 24 columns. The columns are organized in pairs, in 
each pair of columns the first represents the address in the buffer while the second
its content. Each row represents...

\subsubsection{Output spy-buffers dump}
\label{sec:spyout}

The \textbf{\texttt{ambslp\_inp\_spy\_main}}\index{ambslp\_inp\_spy\_main} command

\subsubsection{Spy-buffers dumper}
\label{sec:spydumper}

The \textbf{\texttt{ambslp\_spybuffer\_dumper\_main}}
\index{ambslp\_spybuffer\_dumper\_main}
 command includes
the functionalities of the previous two commands, with the advantage of allowing to
read the them in a synchronized way. Additional features are also available, as
the ability to store the information directly in files that can be used as input
for the board simulation or as input for a standalone test of the board, see
\ref{sec:boardsim}.