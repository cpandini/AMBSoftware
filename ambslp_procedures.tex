The tools describes up to now usually provide a low-level
interaction with the AM board, controlling a single feature
or asserting a single command at the time. It is however 
possible to perform high level procedures, composed by a sequence
of commands.

To perform this kind of operation the command line tool
\textbf{\texttt{\index{ambslp\_procedure}}} commands can be used. The
command line to run this tool will be similar to the
following:
\begin{verbatim}
$ ambslp_procedure [options] transition1 [transition2 ...]
\end{verbatim}
where the options block can specify parameters used by one, or more,
of the following \emph{transitions}. The optiosn are often in common
with other tools, already described, clarifications already by the emended help.
The \emph{transitions} are commands related to common procedures
that the board needs, the name are often in common with the
run control module. The transitions will be executed
sequentially, following the order in the submission string, they can 
be also executed multiple times.

Available transitions are described in deatails within the following 
subsections. The name of the procedures are often related to the run
control transition, more details in \ref{sec:rcmodule}.

\subsection{Configure the boards}
\label{sec:procconfigure}

The procedure keywords are \textbf{\index{ambslp\_procedure!configure}} or \textbf{config}  this procedure provide some basic configuration of the boards, should done every time a test start. It
    also configure the AM chips output and the GTP link from and to the
    ROAD FPGA.

\subsection{Loading a pattern bank}
\label{sec:procloadbank}
The keywords are \textbf{loadbank} and \textbf{load}, 
this command performs the
	sequence of operations required to write a pattern bank within
	the AM chips in the board. The parameters for the pattern bank 
	that will be written are decided through command line options.
	Used options: \texttt{bankpath}, \texttt{type}, \texttt{N}.
    
    The \texttt{bankpath} option allows to set the pattern bank
    file to be loaded. The path can be a special keyword that
    allows automatic bank generation, two types of banks are
    currently supported: \texttt{random}, generating a random bank,
    \texttt{seq[N]}, generating bank where all super-strip of
    a given pattern are equal and function of the pattern position
    of the chip following this formula: $SS[i] = i\%N$, where \% is
    the integer rest operator, i the pattern position within the chip
    and N the optional parameter passed after the \texttt{seq}
    keyword, if not specified this is equal to the number of patterns
    per chip loaded. \texttt{N}, or \texttt{nPattsPerChip}, sets
    the number of patterns per chip that will be loaded.

\subsection{Connect the boards}
\label{sec:procconnect}

The keyword is	\textbf{connect}, this connects transition
	of the ATLAS run control software, more explanation later, and
	performs the link synchronization for the chips' input busses and
	with the HIT FPGAs, also ensure the input links from the AUX
	are properly set.
