\label{sec:boardsim}
A small program emulating the board is available within the software
package. The board simulation program is 
\textbf{\texttt{ambslp\_boardsim}}\index{ambslp\_boardsim}. 
The goal of the program 
is compare a given input with the content of a pattern bank, as if loaded
by the real board, and predict list of the output roads.

An minimal example of how to use the simulation program is:
\begin{verbatim}
$ ambslp_boardsim -T 0 -N 131072 patterns.pbank.root sshits__L{0..11}.ss
\end{verbatim}
the first options are used to set the pattern bank type and how many
patterns should be loaded per chip, in this case all patterns in an AM06 chip.
The following arguments are the pattern bank file, in the same format
used to populate the chips of a real board, while the following files are
the list of input super-strips, here 12 files, 1 for each input link.
The simulation will save the list of output roads in a file named, by default,
\texttt{outroads\_sim.out}, having the same format of the output 
spy-buffer dump using method 2.